\chapter{Agentic Pipeline Developoment}
\subsection{Problem Understanding}
Legal consultation for gender-based cases involves a complex interplay of factual assessment, 
legal analysis, and interpersonal support. To understand the challenges our agentic pipeline 
aims to address, it is essential to examine the current process of legal consultation in the 
Colombian context, particularly within institutions like the Consultorio Jurídico de la Universidad de los Andes.

When conducting initial consultations, legal experts engage in a structured yet flexible 
information-gathering process. This typically begins with open-ended questions about the 
client's situation, followed by increasingly specific inquiries to establish relevant facts, 
timelines, and supporting documentation. Throughout this process, legal practitioners must 
balance multiple objectives: identifying legally relevant details, establishing case chronology, 
determining applicable legal frameworks, assessing procedural options, and evaluating evidentiary 
strengths and weaknesses.

In the specific context of gender-based cases, this process becomes particularly nuanced. 
Legal experts must navigate sensitive topics such as domestic violence, sexual harassment, 
discrimination, and family conflicts while maintaining professional boundaries and creating 
a safe environment for disclosure. The consultation process often requires practitioners to 
recognize both explicit and implicit elements of gender-based issues, including power dynamics, 
economic dependencies, and potential safety concerns that may not be immediately articulated by clients.

At the Consultorio Jurídico de la Universidad de los Andes, consultations serve a dual purpose. 
Beyond providing legal guidance, practitioners offer crucial emotional support to clients navigating 
traumatic or stressful situations. This holistic approach recognizes that legal challenges do 
not exist in isolation from psychological and social contexts. Practitioners are trained to 
acknowledge emotional aspects of cases while guiding clients toward practical legal 
solutions—a balance that any automated system must similarly attempt to achieve.

The Consultorio Jurídico operates as a social service initiative, providing free legal assistance 
to populations that cannot afford private legal representation. This service, typically costly 
in the commercial sector, forms part of the university's commitment to social responsibility 
and access to justice. Law students, supervised by experienced attorneys, gain practical 
experience while serving vulnerable communities, creating a mutual benefit model that contributes 
to both legal education and social equity.

A significant challenge in developing technological support for this process is data access. 
The Consultorio Jurídico maintains detailed records of consultations, including case notes, 
legal analyses, and outcomes. However, these records contain sensitive personal information 
and are not anonymized, making them unsuitable for sharing with third-party researchers or 
technology developers. Client confidentiality and ethical considerations rightfully prevent 
direct use of these authentic case records for AI development purposes.

This data limitation creates a methodological challenge for our research. Without access to 
real consultation transcripts or case files, we must develop alternative approaches to creating 
representative datasets for training and evaluating our agentic pipeline. This necessitates 
collaboration with legal experts to generate synthetic cases that accurately reflect the patterns, 
complexities, and legal issues commonly encountered in gender-based consultations, while ensuring 
no actual client information is compromised.
Additionally, the consultation process varies significantly based on the specific legal issues 
involved. A domestic violence consultation may focus heavily on immediate protective 
measures and evidence documentation, while a workplace discrimination case might center on 
establishing patterns of differential treatment and identifying appropriate administrative and 
judicial remedies. This diversity of case types requires any automated system to be flexible enough 
to adapt its information-gathering approach based on the emerging understanding of the case category.
The temporal dimension of legal consultations also presents challenges for automation. 

Initial consultations often reveal only partial information, with critical details emerging over 
subsequent interactions as trust is established and clients recall or discover relevant information. 
This iterative process must be accommodated in any agentic system, allowing for recursive 
information gathering and strategy refinement as case understanding evolves.
Finally, the legal consultation process culminates in strategy formation—identifying viable 
legal pathways, explaining procedural requirements, outlining potential challenges, and setting 
realistic expectations about timelines and outcomes. This strategic guidance must balance 
technical accuracy with practical feasibility, considering not just what is legally possible 
but what is realistically achievable given the client's resources, priorities, and circumstances.
Understanding these multifaceted aspects of legal consultation is essential for developing an 
effective agentic pipeline. Our system must not only extract relevant information and provide 
accurate legal guidance but must do so in a manner that acknowledges the emotional, ethical, 
and practical dimensions of gender-based legal challenges in the Colombian context.

\section{Dataset Construction}
\label{sec:dataset}

Our dataset development followed an iterative human-AI collaboration process comprising four key phases, as illustrated in Figure~\ref{fig:data_pipeline}.

\subsection{Base Case Preparation}
\begin{itemize}
    \item Initialized with 23 anonymized gender-based case summaries from the Consultorio Jurídico
    \item Categorized cases into primary themes including:
    \begin{itemize}
        \item Violencia Basada en Género
        \item Violencia Sexual
        \item Violencia en el Ámbito Familiar
        \item Violencia en el Ámbito Laboral
        \item Interseccionalidad
    \end{itemize}
\end{itemize}

\subsection{Synthetic Case Generation}
\begin{enumerate}
    \item For each base case, prompted GPT-4o with:
    \begin{itemize}
        \item Case category
        \item Example template from similar cases
        \item Constraints on demographic details
    \end{itemize}
    
    \item Generated [M] cases, randomizing case category.
\end{enumerate}

\subsection{Expert-Guided Refinement}
\begin{itemize}
    \item Legal practitioners ([K] law students, [L] attorneys) performed:
    \begin{enumerate}
        \item Case completeness verification
        \item Development of question-answer pairs:
        \begin{itemize}
            \item 10-20 diagnostic questions per case
            \item Culturally-adapted follow-up probes
        \end{itemize}
        
        \item Structural edits to ensure procedural accuracy
    \end{enumerate}
\end{itemize}

\subsection{Conversation Simulation}
\begin{itemize}
    \item Generated lawyer-client dialogues through iterative prompt refinement:
    \begin{equation}
        C_{final} = \text{LLM}(P_2(\text{LLM}(P_1(C_{base})))
    \end{equation}
    where:
    \begin{itemize}
        \item $C_{base}$ = initial synthetic case
        \item $P_1$ = first prompt template + initial context
        \item $P_2$ = first prompt template + expert provided context
    \end{itemize}
    
    \item Implemented feedback-driven generation cycle:
    \begin{enumerate}
        \item Initial generation: Created dialogues using base prompt template
        \item Expert analysis: Added [P] critical questions, removed [Q] irrelevant ones
        \item Final generation: Regenerated conversations using refined prompt template
    \end{enumerate}
\end{itemize}

\subsection{Legal Strategy Development}
\begin{itemize}
    \item LLM-produced strategies conditioned on:
    \begin{itemize}
        \item Full conversation history and generated legal case
        \item Relevant Colombian legal articles
        \item Regional International legal articles
    \end{itemize}
    
    Legal experts then validated the output data, ensuring we have good
    examples to train our models on.
\end{itemize}

\subsection{Final Dataset Composition}
\begin{table}[h]
    \centering
    \caption{Dataset Statistics}
    \label{tab:dataset_stats}
    \begin{tabular}{lc}
        \toprule
        \textbf{Component} & \textbf{Count} \\
        \midrule
        Base Cases & [N] \\
        Synthetic Cases & [M] \\
        Curated Dialogues & [P] \\
        Legal Strategies & [Q] \\
        Expert Annotations & [R] \\
        \bottomrule
    \end{tabular}
\end{table}

ANALYSIS!!!
ANALYSIS!!!
ANALYSIS!!!
\section{System Architecture}
\label{sec:architecture}

Our agentic pipeline employs a three-stage architecture optimized for legal case processing, as shown in Figure~\ref{fig:system_flow}. Each component operates sequentially with validation checks between stages.

\subsection{Information Extraction Module}
\begin{itemize}
    \item \textbf{Conversational Agent}:
    \begin{itemize}
        \item Conducts dialogic interaction with users through natural language
        \item Implements context-aware questioning based on:
        \begin{itemize}
            \item Emerging case details
            \item Predefined legal factor templates
            \item User response patterns
        \end{itemize}
    \end{itemize}
    
    \item \textbf{Completion Checker}:
    \begin{itemize}
        \item Monitors conversation stream in real-time
        \item Evaluates information sufficiency using either the model itself, 
        along with prompt engineering or a finetuned model, 
        trained using the dataset developed in the previous setion:
        \item Triggers phase transition when the model answer's "yes"
    \end{itemize}
\end{itemize}

\subsection{Legal Case Generation}
\begin{itemize}
    \item Input transformation process:
    \begin{itemize}
        \item Filters conversational data using completeness checker output
        \item Structures information into formal legal elements:
        \begin{itemize}
            \item Parties identification
            \item Case type classification
            \item Chronological event timeline
            \item Evidence inventory
            \item Client petitions
        \end{itemize}
    \end{itemize}
\end{itemize}

\subsection{Legal Strategy Generation}
\begin{itemize}
    \item \textbf{RAG Architecture}:
    \begin{itemize}
        \item Knowledge Base: Colombian legal corpus containing:
        \begin{itemize}
            \item International procedural codes
            \item Regional procedural codes
        \end{itemize}
        
        \item Retrieval Component:
        \begin{itemize}
            \item Dense vector search using BAAI Embeddings + HyDE
            \item Statute relevance ranking
        \end{itemize}
        
        \item Generation Component:
        \begin{itemize}
            \item Contextual fusion of case details + legal provisions
            \item Strategy template instantiation
        \end{itemize}
    \end{itemize}
\end{itemize}

\begin{table}[h]
    \centering
    \caption{Architecture Components}
    \label{tab:components}
    \begin{tabular}{lp{0.7\textwidth}}
        \toprule
        \textbf{Module} & \textbf{Function} \\
        \midrule
        Information Extraction & Dialogic fact gathering with completeness verification \\
        Case Generation & Conversation-to-legal-document transformation \\
        Strategy Generation & RAG-based legal analysis and recommendation \\
        \bottomrule
    \end{tabular}
\end{table}

The pipeline enforces strict data isolation between stages, 
with case information only progressing upward when meeting 
validation criteria from the previous phase.
