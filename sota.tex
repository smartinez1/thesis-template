\chapter{State of the Art}
\label{cha:sota}

%%
% \section{Top level section}

% %%%%
% \subsection{Second level section}

% %%%%%%
% \subsubsection{Third level section}

\section{State of the Art}

The rapid advancement of Large Language Models (LLMs) and their 
applications in legal contexts has opened new avenues for making 
legal systems more accessible, particularly for non-expert users. 
In recent years, LLMs have demonstrated their utility in processing 
legal documents, generating advice, and improving legal 
decision-making processes, but challenges remain regarding their 
integration and effectiveness in highly specialized domains such as 
law.
Wu et al. (2023) introduced the Diagnostic Legal Large Language Model (D3LM), 
which addresses one of the key issues faced by non-legal experts: 
the difficulty of formulating accurate legal queries. D3LM employs 
adaptive, lawyer-like diagnostic questions to guide users, ensuring 
that relevant legal factors are captured and considered when interacting with the LLM. 
This model uses a novel Positive-Unlabeled Reinforcement Learning (PURL) approach to 
generate targeted questions, thus improving the feedback quality and user experience. 
By incorporating a Court Views Generation (CVG) dataset, 
D3LM outperforms traditional LLMs in the legal domain by focusing on user interaction 
and precision, making it a crucial advancement in bridging the gap between legal 
professionals and laypeople.

On the other hand, the NOMOS model developed by Pennisi et al. (2023) 
emphasizes the identification of legal obligations within statutes. 
The complexity of legal language often makes it difficult to discern the 
obligations hidden within long, dense legal documents. 
NOMOS uses a hybrid approach combining Positional Embeddings (PE) and 
Temporal Convolutional Networks (TCNs), allowing for more effective 
mining of obligations from legal texts. This method shows how deep 
learning techniques can be utilized to process and extract actionable 
insights from extensive legal documents, further improving the 
accessibility of legal information.

Legal chatbots, as highlighted by Chakrabortyi (2023), 
have emerged as powerful tools in democratizing access to legal services.
These AI-driven chatbots provide general legal advice, assist in drafting 
legal documents, and even help with dispute resolution. By offering 
cost-effective, on-demand legal guidance, chatbots like DoNotPay and 
LegalZoom reduce barriers to legal services for people who may lack 
the resources to hire traditional legal professionals. However, 
while these tools are useful for routine queries, they remain 
supplementary and cannot fully replace the nuanced expertise of human 
lawyers. This demonstrates both the potential and the current 
limitations of AI in delivering justice.

Lastly, Mavi et al. (2023) explore the challenges LLMs face when dealing 
with specialized legal tasks that require extensive background knowledge. 
Their work focuses on the Retrieval-Augmented Chain-of-Thought (CoT) 
model, which leverages the semi-structured nature of legal data to 
retrieve relevant context efficiently. The retrieval system is designed 
to work within the token limitations of current LLMs while ensuring 
that important contextual information is incorporated into the model’s 
reasoning process. This approach enhances the performance of LLMs in 
legal question-answering tasks, showcasing how retrieval-augmented 
methods can significantly improve accuracy and comprehension in 
specialized domains like law.

Together, these studies illustrate the evolving role of AI and LLMs in 
legal systems, highlighting innovations in user interaction, obligation 
extraction, legal chatbots, and context retrieval. Each development 
contributes to a broader vision of how AI can make complex legal processes 
more understandable and accessible, particularly for the general public 
in legal environments like Colombia's, where navigating the legal system 
can be daunting for non-experts. By combining advancements in machine 
learning, natural language processing, and user-centered design, LLMs and 
related AI models are pushing the boundaries of how legal information is 
processed and delivered to diverse audiences.

\endinput