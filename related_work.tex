\chapter{Related Work}
\label{cha:sota}

%%
% \section{Top level section}

% %%%%
% \subsection{Second level section}

% %%%%%%
% \subsubsection{Third level section} 
\section{Referential Framework}
\subsection{Large Language Models}
Large Language Models (LLMs) represent a significant advancement 
in natural language processing, characterized by their ability to 
understand, generate, and interact with human language at unprecedented 
scales. Modern LLMs such as GPT-4~\cite{openai2023}, LLaMA~\cite{grattafiori2024llama3herdmodels}, 
and Claude~\cite{anthropic2023} are trained on vast corpora of text data, 
enabling them to perform a wide range of language tasks without task-specific training. 
These models have demonstrated remarkable capabilities in text generation, summarization, translation, 
and even complex reasoning tasks~\cite{brown2020}.
\subsection{Transformer Architecture}
The transformer architecture, introduced by Vaswani et 
al.~\cite{vaswani2023attentionneed}, forms the foundation of modern 
LLMs. Unlike recurrent neural networks, transformers process entire 
sequences simultaneously through self-attention mechanisms, allowing for 
more efficient training and better modeling of long-range dependencies in 
text. The architecture consists of an encoder and decoder, each composed 
of multi-head attention layers and feed-forward neural networks, enabling 
parallel processing and effective representation learning.
\subsection{Generative and Discriminative LLMs}
LLMs can be categorized as either generative or discriminative based on
their primary function. Generative LLMs, such as GPT models, focus on 
producing text by predicting the next token in a sequence given previous 
tokens. These models excel at creative writing, content generation, and
open-ended dialogue. Discriminative LLMs, in contrast, classify or 
categorize text into predefined classes or extract specific information 
from text, making them suitable for sentiment analysis, entity recognition,
and other classification tasks.
\subsection{Encoder-only, Decoder-only, and Encoder-Decoder LLMs}
The architectural design of LLMs can be categorized into three primary 
types: encoder-only, decoder-only, and encoder-decoder models.
Encoder-only models, such as BERT~\cite{devlin2019}, excel at understanding
and representing input text for classification and information extraction 
tasks. Decoder-only models, including GPT~\cite{brown2020} and 
LLaMA~\cite{grattafiori2024llama3herdmodels}, specialize in text 
generation. Encoder-decoder models, like T5~\cite{ni2021sentencet5scalablesentenceencoders} and 
BART~\cite{lewis2019bartdenoisingsequencetosequencepretraining}, combine 
both components to process input text and generate output sequences, 
making them versatile for translation, summarization, and question-answering tasks.
\subsection{Pre-training and Fine-tuning Paradigm}
Modern LLMs typically follow a two-stage development process: pre-training 
and fine-tuning. During pre-training, models learn general language 
understanding from vast unlabeled text corpora using self-supervised 
objectives such as masked language modeling or next-token prediction. 
This phase creates a foundation model with broad language capabilities. 
Fine-tuning then adapts these pre-trained models to specific downstream 
tasks using smaller, task-specific datasets. This paradigm allows for 
transfer learning, where knowledge gained from pre-training can be 
leveraged for various applications with minimal task-specific 
data~\cite{howard2018}.
\subsection{Supervised Fine-Tuning (SFT) and Reinforcement Learning from Human Feedback (RLHF)}
Supervised Fine-Tuning (SFT) involves training pre-trained models on 
labeled datasets for specific tasks. This approach uses traditional 
supervised learning methods where the model learns from input-output pairs. 
Reinforcement Learning from Human Feedback (RLHF), introduced by Christiano 
et al.\cite{christiano2017} and popularized by InstructGPT\cite{ouyang2022}, 
extends beyond SFT by incorporating human preferences into the training 
process. RLHF typically involves training a reward model based on human 
comparisons of model outputs, then using reinforcement learning to optimize 
the model's behavior according to this reward function. This approach has 
proven effective in aligning LLMs with human values and preferences, 
reducing harmful outputs, and improving helpfulness.
\subsection{Efficient Fine-tuning with LoRA and QLoRA}
As LLMs grow in size, full fine-tuning becomes computationally prohibitive. 
Low-Rank Adaptation (LoRA), introduced by Hu et al.\cite{hu2021}, offers 
an efficient alternative by freezing the pre-trained model weights and 
injecting trainable low-rank decomposition matrices into each layer of 
the transformer architecture. This approach significantly reduces the 
number of trainable parameters while maintaining performance. Quantized 
Low-Rank Adaptation (QLoRA), developed by Dettmers et al.\cite{dettmers2023}, 
further improves efficiency by combining LoRA with model quantization techniques, 
enabling fine-tuning of models with billions of parameters on consumer-grade 
hardware without sacrificing performance.
\subsection{Retrieval Augmented Generation - RAG}
Retrieval Augmented Generation (RAG), proposed by Lewis et 
al.\cite{lewis2020retrieval}, combines information retrieval with text 
generation to enhance the factuality and reliability of LLM outputs. 
RAG systems first retrieve relevant documents from an external knowledge 
base using the input query, then condition the language model's generation 
on both the input query and the retrieved documents. This approach allows 
LLMs to access information beyond their training data, reducing 
hallucinations and enabling them to cite sources explicitly. 
RAG has proven particularly valuable in knowledge-intensive domains 
such as legal analysis, where accurate reference to authoritative 
sources is essential.
\subsection{Hypothetical Document Embeddings - HyDE}
Hypothetical Document Embeddings (HyDE), introduced by Gao et al.\cite{gao2022}, 
addresses the challenge of retrieval in dense vector spaces when queries 
differ substantially from relevant documents. HyDE uses an LLM to generate a 
hypothetical document that might answer the query, then uses this document's embedding rather 
than the query's embedding for retrieval. This approach bridges the semantic gap between queries 
and documents, improving retrieval performance especially for complex or hypothetical queries. In legal 
applications, HyDE can help identify relevant precedents or regulations that may not share obvious lexical 
similarities with a user's query\cite{gao2022}.
\subsection{Prompt Engineering}
Prompt engineering encompasses techniques for formulating inputs to LLMs to 
elicit desired outputs. Effective prompts can significantly enhance model 
performance without changing model parameters. Key techniques include few-shot 
learning, where examples are provided within the prompt; chain-of-thought prompting, 
which encourages step-by-step reasoning; and system prompts that define the model's role 
and constraints. In legal applications, prompt engineering can guide models to adhere 
to specific frameworks, consider relevant factors, and produce outputs aligned with 
legal standards and practices.
\subsection{Agentic Frameworks for LLMs}
Agentic frameworks extend LLMs beyond passive response generation to enable goal-directed, 
multi-step reasoning and action. These frameworks typically involve decomposing complex 
tasks into subtasks, planning execution sequences, and monitoring progress toward goals. 
Approaches like ReAct integrate reasoning and action, while frameworks such
as AutoGPT and LangChain provide structures for autonomous task completion. 
In legal contexts, agentic frameworks can orchestrate the process of case analysis, 
from information gathering to strategy formulation and explanation.
\subsection{Evaluation Metrics for LLMs}
Evaluating LLM performance requires multiple complementary metrics. General metrics 
include perplexity, which measures how well a model predicts a sample, 
and BLEU score for comparing generated text to reference texts. 
BERTScore~\cite{zhang2020bertscoreevaluatingtextgeneration}, which uses contextual 
embeddings to measure semantic similarity between generations and references, 
offers a more nuanced evaluation of text quality. For legal applications, 
domain-specific metrics might include assessments of legal accuracy, 
procedural correctness, and citation validity. Human evaluation remains crucial, 
particularly for assessing qualities like helpfulness, clarity, and appropriateness 
of legal guidance.
\subsection{Hallucinations and Bias}
LLMs are prone to hallucinations—generating content that is factually incorrect 
or unfounded—and may perpetuate or amplify biases present in their training data. 
Hallucinations pose particular challenges in legal contexts, where accuracy is 
paramount. Techniques to mitigate hallucinations include retrieval augmentation, 
which grounds generations in external knowledge sources, and uncertainty 
quantification, which helps identify when a model might be generating unreliable 
content. Addressing bias requires careful dataset curation, targeted fine-tuning, 
and ongoing monitoring of model outputs. In legal applications, these considerations 
are critical for ensuring that AI-generated guidance does not propagate systemic 
inequities or provide misleading information.

\section{State of the Art}
\subsection{Datasets}
Many of the datasets regarding legal text are limited to the English and
Chinese speaking world. This limitation extends to other jurisdictions as well, 
creating challenges for developing legal AI systems in countries like Colombia 
where access to structured legal datasets is either restricted or non-existent.
\subsection{Approaches}
The rapid advancement of Large Language Models (LLMs) and their applications 
in legal contexts has opened new avenues for making legal systems more 
accessible, particularly for non-expert users. LLMs have demonstrated 
significant utility in processing legal documents, generating advice, 
and improving legal decision-making processes, yet challenges remain, 
particularly in highly specialized domains such as law. Recent works have 
shown that, while LLMs can assist laypeople with legal tasks, they also 
struggle with issues such as hallucination and the proper sourcing of 
high-quality legal data, as noted by Dahan et al.~\cite{dahan2023lawyers}.
Legal chatbots, as highlighted by Chakrabortyi~\cite{chakraborty2023revolutionizing}, 
have become powerful tools in democratizing access to legal services by 
offering cost-effective, on-demand guidance. These AI-driven systems, 
including well-known platforms such as DoNotPay and LegalZoom, 
help users draft documents and resolve disputes, although they cannot 
fully replace the nuanced expertise of human lawyers. Alongside these, 
works like Wu et al.'s Diagnostic Legal Large Language Model 
(D3LM)~\cite{wu2024knowledgeinfusedlegalwisdomnavigating} focus on improving LLM 
interactions with non-expert users by employing lawyer-like diagnostic 
questions to better guide legal queries. This model's use of 
Positive-Unlabeled Reinforcement Learning (PURL) significantly 
enhances the user experience by generating more relevant, targeted questions, 
ensuring that pertinent legal factors are considered during interactions 
with the LLM.
\subsection{Retrieval-Augmented Approaches}
In contrast to more complex methods like D3LM, Retrieval-Augmented Generation 
(RAG) approaches offer a compelling advantage by requiring less data and 
providing more straightforward processing pipelines. While D3LM utilizes 
diagnostic questions and relies on Positive-Unlabeled Reinforcement Learning 
(PURL) to guide legal interactions, its process is notably intricate, 
involving knowledge graphs that are highly specific to particular 
legal domains. This presents a limitation when the field of expertise changes, 
as the knowledge graph would need to be adjusted or rebuilt for different legal 
areas.
RAG, on the other hand, bypasses this complexity by focusing on retrieving 
minimal, highly relevant text segments from large legal corpora, as 
demonstrated by LegalBench-RAG \cite{pipitone2024legalbenchragbenchmarkretrievalaugmentedgeneration}. 
By emphasizing precise retrieval and enabling LLMs to generate accurate 
citations, RAG methods allow for efficient, domain-agnostic solutions that 
maintain accuracy without the overhead of domain-specific modeling. Recent 
research by Manathunga et al. \cite{manathunga2023retrievalaugmentedgenerationrepresentative} has 
further demonstrated RAG's potential in enhancing legal text summarization by dynamically 
fetching relevant documents before generating responses. Similarly, 
Lee et al. \cite{ryu-etal-2023-retrieval} explored RAG's application in case 
law retrieval, showing its superiority over traditional keyword-based search 
engines.
The integration of FAISS (Facebook AI Similarity Search) with RAG models has 
significantly improved document retrieval efficiency in legal applications. 
\cite{panchal2025lawpalretrievalaugmented} the authors demonstrated that FAISS-based vector 
search mechanisms outperform conventional database searches in legal 
information retrieval, reducing query response time while maintaining high accuracy. 
In a comparative study, \cite{zeng2023scalableeffectivegenerativeinformation}, FAISS-based 
retrieval mechanisms significantly outperformed traditional Boolean keyword searches, 
reducing irrelevant document retrieval by 40\%.
\subsection{Transformer-Based Models and Deep Learning}
In another vein, the NOMOS model by Pennisi et al. 
\cite{pennisi-etal-2023-nomos} focuses on the identification of legal 
obligations within statutes, addressing the complexity of legal language and 
enabling a more efficient extraction of actionable insights from dense legal 
texts. By combining Positional Embeddings (PE) with Temporal Convolutional 
Networks (TCNs), NOMOS demonstrates how deep learning can streamline the 
process of mining obligations from legal documents, making legal information 
more accessible.
Transformer-based models such as BERT and GPT have contributed significantly 
to AI-driven legal research. RoBERTa, an optimized version of BERT developed 
by \cite{Lliu2019robertarobustlyoptimizedbert}, has enhanced contextual understanding 
and document similarity matching in legal queries. These models have been 
integrated into legal chatbots for contract analysis and legal decision-making, 
improving legal text comprehension and summarization.
\subsection{Specialized Legal Question-Answering Systems}
Meanwhile, Jonathan Li et al. \cite{li2024experimentinglegalaisolutions} 
introduce a human-centric approach to legal question-answering systems in 
their work, which develops the LegalQA dataset. This dataset comprises real, 
expert-curated legal questions and answers spanning diverse legal areas. 
Their research emphasizes the use of structured data and retrieval-augmented 
generation to boost LLM performance, particularly in producing factually 
correct and comprehensible answers for laypeople. This framework is an 
important step towards addressing some of the gaps found in general-purpose LLMs, 
particularly by ensuring that the legal information delivered is both accurate and 
relevant to users' needs.
\subsection{Chain-of-Thought and Reasoning Approaches}
Mavi et al. \cite{mavi2023retrievalaugmentedchainofthoughtsemistructureddomains} 
explore the Retrieval-Augmented Chain-of-Thought (CoT) model, which tackles specialized 
legal tasks by efficiently retrieving context from semi-structured data. Their model 
enhances the reasoning abilities of LLMs, especially in legal question-answering 
scenarios, by incorporating important contextual information within token limitations.
While the Chain-of-Thought (CoT) approach offers valuable step-by-step reasoning that 
improves the decision-making process in complex legal tasks, it comes with its own challenges. 
CoT relies heavily on structured reasoning, which is well-suited for certain specialized legal tasks 
but can increase the complexity of model outputs. This added complexity, especially when dealing with 
semi-structured data, can sometimes hinder scalability and efficiency when compared to the more straightforward 
retrieval and generation process of RAG. Thus, while CoT has clear benefits in enhancing accuracy for complex 
legal queries, RAG's simplicity and adaptability make it particularly advantageous for real-world applications 
where legal fields and contexts are constantly shifting.
\subsection{Addressing Hallucinations and Evaluation}
One of the most prominent problems when it comes to LLMs, especially in the legal domain, 
is the presence of hallucinations. For instance, when replying to a legal question, 
the model may come up with non-existent laws or contradictory recommendations. 
It is of very high importance then, to avoid this kind of issues. 
In \cite{hu2025finetuninglargelanguagemodels} the authors create a benchmark 
to evaluate model's hallucinations, using the groundtruth law set as a reference.
\cite{Papineni_bleu} and \cite{lin-2004-rouge} introduced automated metrics 
such as BLEU and ROUGE scores which can be used to evaluate AI-generated 
legal text summaries, ensuring their quality and relevance.
\subsection{Access to Justice and Ethical Considerations}
The practical applications of legal AI chatbots have been studied extensively 
in the context of access to justice and AI ethics. 
[citation!!!!] highlights the potential of AI-driven legal assistants in bridging 
the justice gap, particularly in countries where legal resources are not easily 
accessible. This is particularly relevant for systems like Colombia's, 
where legal access disparities can be significant. 
Research by \cite{Min_bias} explored methods for bias detection and mitigation 
in legal AI, ensuring fairness in AI-generated legal advice.
\subsection{Challenges and Future Directions}
Despite these advancements, challenges remain in AI-driven legal research. 
Existing chatbots still struggle with multi-jurisdictional legal queries. 
Additionally, legal AI models often lack the ability to process long-context 
legal arguments effectively, a limitation that has prompted proposals for 
memory-based retrieval techniques to improve long-form legal text processing.
This combination of streamlined retrieval and adaptability makes RAG methods 
particularly suited to dynamic legal environments, such as Colombia's. where 
legal systems can be difficult for non-experts to navigate. 
\cite{Singh_legal} demonstrated that AI-powered legal research tools using 
NLP provide faster and more contextually accurate responses compared to standard legal 
databases, further supporting the potential for RAG-based systems in making 
complex legal systems more accessible.
% \subsection{Domain-Specific LLMs for Legal and Regulatory Text}
% In \cite{martinez-etal-2025-scalable}, where we present a comprehensive approach to 
% developing LLMs tailored specifically for understanding regulatory texts. 
% Our methodology involves constructing a specialized corpus through 
% large-scale scraping of financial and regulatory documents across domains 
% like compliance, licensing, and financial reporting. By preprocessing this 
% data with GPT-4o-mini and further pre-training a LLaMA-3.1-8B model on the 
% curated corpus, we demonstrate improved capability in tasks such as acronym 
% expansion and regulatory question-answering. Our implementation of 
% Quantized Low-Rank Adaptation (QLoRA) addresses computational 
% efficiency concerns while maintaining model performance. 
% Although our model shows only slight improvements over baseline in some tasks, 
% particularly in named entity recognition, this approach illustrates the 
% potential of domain-specific LLMs in regulatory text interpretation and 
% establishes a foundation for specialized NLP evaluation methodologies. 
% This scalable framework offers a balanced alternative between computational 
% requirements and performance, allowing potential deployment on more accessible 
% hardware compared to larger models, making it particularly relevant for contexts 
% where computational resources may be limited, such as Colombia's legal system.
% Future developments in multilingual legal AI, enhanced retrieval mechanisms, and 
% AI-powered contract analysis will be crucial in making legal AI tools more accessible, 
% reliable, and widely applicable in legal practice, particularly in diverse jurisdictions 
% like Colombia where multiple language variations and regional legal nuances may exist.
\endinput

