\chapter{Introduction}
\label{cha:introduction}

%%
% \section{Top level section}

% %%%%
% \subsection{Second level section}

% %%%%%%
% \subsubsection{Third level section}
\subsection{Context} 
El sistema legal colombiano enfrenta tres desafíos estructurales:
\begin{itemize}
    \item \textbf{Complejidad procesal}: Requisitos burocráticos excesivos para trámites comunes (e.g., [Ejemplo específico de proceso legal])
    \item \textbf{Disparidades geográficas}: Disminución progresiva de abogados/jueces por habitante en zonas rurales vs. urbanas
    \item \textbf{Desconfianza institucional}: [Citación a estudio cualitativo sobre percepción ciudadana]
\end{itemize}

El Consultorio Jurídico de la Universidad de los Andes ha documentado [N] casos de mujeres entre [Año]-[Año], proporcionando insights valiosos sobre:
\begin{itemize}
    \item Tipologías de casos recurrentes (violencia doméstica, derechos de custodia, etc.)
    \item Barreras específicas en el acceso a la justicia
\end{itemize}
\subsection{Problem Statement} 
\subsection{Justification} 
\subsection{Objectives} 
Access to justice remains a critical challenge in Colombia, where systemic barriers—including 
opaque legal procedures, bureaucratic inefficiencies, and a lack of affordable legal 
representation—disproportionately affect marginalized populations. 

% According to the Colombian National Administrative Department of Statistics (DANE, 2022), 
% over 60\% of citizens facing legal disputes report inadequate understanding of their rights 
% or the steps required to resolve their cases.  %TODO use actual sources 

This gap in legal literacy exacerbates inequities, 
particularly for vulnerable groups such as women, who often confront intersecting socioeconomic 
and gender-based hurdles when navigating the judicial system.

The Consultorio Jurídico of Universidad de los Andes exemplifies a targeted effort to address these 
disparities. By curating anonymized legal cases involving women—categorized by thematic nature 
(e.g., domestic violence, labor discrimination, family law) and enriched with annotations 
from legal professionals—the institution has established a structured, ethically compliant repository. 
This dataset not only reflects prevalent legal challenges faced by women in Colombia but also provides a rigorously validated foundation for developing AI-driven tools tailored to real-world needs.

Focusing on this subset of cases offers strategic advantages. First, it narrows the scope to a high-impact, well-defined problem space, enabling precise evaluation of the tool’s efficacy in scenarios where clarity and accuracy are paramount. Second, success in this context would validate the scalability of the proposed architecture: by addressing nuanced, jurisdiction-specific cases initially, the system can later generalize to broader legal domains, such as tenant rights or consumer protection, while maintaining compliance with Colombia’s civil law framework.

Ultimately, this work seeks to democratize legal empowerment by bridging the gap between complex institutional processes and public comprehension. A tool capable of guiding users through their legal options—clarifying whom to consult, which procedures to follow, and what rights apply—has the potential to reduce reliance overburdened legal aid services, mitigate systemic inequities, and foster trust in judicial institutions. By leveraging AI to translate legalese into actionable insights, this initiative aligns with global Sustainable Development Goals (SDGs), notably SDG 16: Peace, Justice, and Strong Institutions.

\endinput