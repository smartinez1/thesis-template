\chapter{Introduction}
\label{cha:introduction}

%%
% \section{Top level section}

% %%%%
% \subsection{Second level section}

% %%%%%%
% \subsubsection{Third level section}
\section{Context}
Recent advances in Large Language Models (LLMs) like GPT-4~\cite{openai2023},
LLaMA~\cite{grattafiori2024llama3herdmodels}, and 
Deepseek~\cite{deepseekai2025deepseekr1incentivizingreasoningcapability} have revolutionized natural
language processing through transformer architectures~\cite{vaswani2023attentionneed}.
Trained on trillions of tokens with up to billions of parameters~\cite{brown2020},
these models excel at text generation, summarization, and complex reasoning tasks.
Their application has expanded into specialized domains including law~\cite{nay2023},
where they demonstrate potential for legal document analysis and
contract review~\cite{chalkidis2022}.
In Colombia, access to justice remains a significant challenge. 
According to a 2021 study by the Ministry of Justice, the National Statistics Department (DANE),
and the National Planning Department (DNP), 56\% of Colombians did not take any action 
to obtain justice for situations related to crimes, poor healthcare, public services, 
housing issues, and family conflicts~\cite{dnp2021}. 
Of those who did seek assistance, 43.1\% approached public institutions or private attorneys,
while 0.9\% resorted to illegal or violent means to address their problems. Overall, 
only 19.8\% of the reported legal needs were satisfied, 
leaving 80.2\% of cases unresolved~\cite{dnp2021}.
The survey, conducted between August and October 2020, included 129,709 individuals from 40,368 
households across Colombia's 13 major cities. 
The study revealed that 83\% of all declared legal needs fell into five categories: 
crimes (3,695,056), healthcare (714,649), public services (407,063), housing (311,130), and family matters (278,510). 
Across all surveyed cities, the most frequently reported legal needs related to public utilities consumption, 
consumer purchases, customer service issues, and family problems—including alimony demands, 
inheritance disputes, and domestic violence. Cali, Pasto, and Ibagué were identified as the cities with the highest number 
of declared legal needs associated with crimes, complaints about healthcare services, and housing issues~\cite{dnp2021}.
The study also found that Colombians primarily seek solutions to their legal needs through institutions such as the police, 
public utility companies, and family commissioner offices.
\section{Problem Statement}
Despite ongoing digitalization efforts in Colombia's legal sector,
a substantial technological gap exists in providing accessible 
legal guidance for gender-based cases. While 56\% of Colombians 
take no action to address their legal needs~\cite{dnp2021}, 
this percentage is even higher among women facing gender-based legal 
issues due to compounding factors: limited legal literacy, financial 
constraints, geographic barriers, and systemic gender biases within 
the justice system.
The complexity of legal frameworks and procedures creates significant 
obstacles for women seeking legal remedies. Traditional approaches to 
legal assistance—including public defenders, legal aid clinics, 
and pro bono services—while valuable, remain insufficient to meet demand. 
This insufficiency is particularly acute in cases involving gender-based violence, 
family law disputes, workplace discrimination, and property rights conflicts, 
where timely intervention is often critical.
Current technological solutions for legal assistance in Colombia 
primarily focus on information dissemination rather than personalized 
case assessment. Existing platforms typically offer static legal 
information or simple document automation but fail to provide the dialogic,
adaptive guidance necessary for complex gender-based cases that require 
nuanced understanding of both factual circumstances and applicable 
laws.

This research addresses this gap by developing an agentic pipeline 
that leverages large language models to create a conversational 
interface specifically designed for gender-based legal cases. 
Unlike conventional legal tech solutions, our system:
\begin{enumerate}
\item Extracts case-specific information through structured yet natural dialogue, accommodating users with varying levels of legal knowledge;
\item Analyzes the collected information within the framework of Colombian law to formulate viable legal strategies;
\item Communicates these strategies in accessible language that balances legal precision with comprehensibility;
\item Focuses specifically on gender-based legal matters, where specialized knowledge of both substantive law and procedural nuances is particularly valuable.
\end{enumerate}
%By narrowing our scope to gender-based legal issues, 
%we confront both a pressing social need and a well-defined 
%technical challenge. The domain specificity enables more 
%rigorous evaluation of the system's accuracy, relevance, 
%and ethical compliance while addressing cases where the 
%consequences of inadequate legal guidance are particularly severe. 
%This focused approach 
Focusing on gender-related cases only also allows for the development of specialized capabilities, 
such as sensitivity to trauma narratives and recognition of patterns specific to 
gender-based discrimination or violence, which might be overlooked in more generalized 
legal assistant systems.
The technical challenge lies in creating an agentic system 
that can effectively navigate between information extraction, 
legal analysis, and accessible communication—all while 
maintaining compliance with ethical guidelines for AI in 
legal contexts and accommodating the varied 
linguistic and educational backgrounds of potential users.
\section{Justification}

From a social perspective, gender-based violence remains a 
critical issue. In 2024, the Instituto Nacional de Salud (INS) reported 66,621 cases of gender-based violence, with 75.6\% of victims being women, amounting to 50,374 cases~\cite{ins2024}. 
This highlights the urgent need for accessible legal assistance, 
especially in rural and low-income communities where access to 
legal professionals is limited.

Technologically, recent advances in large language models (LLMs) 
have created unprecedented opportunities for developing conversational 
interfaces capable of domain-specific reasoning. Modern LLMs can 
process natural language input, maintain contextual awareness 
throughout multi-turn conversations, and generate nuanced 
responses that account for the complexity of legal 
situations~\cite{darrow2023}. This technological maturity 
enables our proposed agentic pipeline to move beyond simple 
information retrieval to true case assessment and strategy 
formulation—capabilities previously exclusive to human legal 
professionals.

Institutionally, Colombia has implemented systems like the 
Integrated Information System on Gender-Based Violence (SIVIGE), 
which reported 58,614 cases of physical violence, 27,585 cases of 
sexual violence, and 10,021 cases of psychological violence in 
2021~\cite{advocates2023}. However, there remains a gap in 
providing intelligent, interactive guidance for case assessment 
and preparation. Our system aims to complement existing resources 
by offering scalable assistance to users who may not have access 
to traditional legal support structures.

From an academic perspective, this work contributes to the 
emerging field of AI for social justice by addressing the following 
open research questions:

\begin{itemize}
\item How can agentic systems effectively extract legally relevant information from non-experts using natural language interaction?
\item What architectural approaches best balance the need for legal precision with accessibility for users with limited legal literacy?
\item What evaluation frameworks most effectively measure the impact of AI legal assistance on access to justice outcomes?
\end{itemize}

%Our focus on gender-based cases provides a methodologically sound 
%framework for addressing these questions. Such cases typically 
%involve well-defined legal domains (family law, labor law, criminal 
%law related to gender violence) with established procedural 
%pathways, allowing for rigorous evaluation of the system’s 
%accuracy and effectiveness. 
Our focus on gender-based issues provides a framework for addressing these questions. The cases in question involve well-defined legal domains which allow for evaluation of the system. 
Additionally, gender-based cases 
often involve complex intersections of factual circumstances 
and legal principles, providing a 
%test bed 
challenging scenario for evaluating 
the system’s capabilities.

Ethically, our approach emphasizes responsible AI development by:

\begin{enumerate}
\item Working with anonymized, ethically sourced case data from relevant legal institutions;
\item Implementing clear disclosure of the system’s limitations and its role as a complement to, not a replacement for, licensed legal counsel;
\end{enumerate}

%By focusing specifically on gender-based legal cases, this research 
%addresses a pressing social need while advancing the technical state 
%of the art in conversational legal AI. Success in this domain would 
%not only provide immediate benefits to a vulnerable population but also 
%establish a methodological framework applicable to other areas of legal 
%assistance, contributing to broader efforts to democratize access to justice.

Even though this work focuses on gender-related legal cases, the same methodology could be applied to other areas in law. This would help provide support the current justice system to provide a more broad access to justice.
\section{Objectives}

\subsection{General Objective}
Develop an agentic pipeline using Large Language Models to provide accessible legal guidance for gender-based cases in Colombia.

\subsection{Specific Objectives}
\begin{enumerate}
    \item Design and implement a conversational agent for extracting comprehensive legal case information through natural dialogue, optimized for gender-based legal matters in the Colombian context.
    
    \item Create a structured knowledge base of Colombian legal documents with document embeddings developed in collaboration with legal experts specializing in gender justice.
    
    \item Implement a Retrieval-Augmented Generation (RAG) system that identifies applicable legal frameworks and formulates viable legal strategies based on given sources.
    
    \item Develop a communication component that tries to translate complex legal strategies into accessible language for users with varying levels of legal literacy.
    
    \item Evaluate system performance using quantitative metrics (Bert-Score for case summaries against ground truth) and qualitative assessment by legal experts, including relevance of extracted documents.
\end{enumerate}

%\section{Scope}
%    \begin{itemize}
%        \item Quantitative assessment of the information extraction agent's accuracy, completeness, and efficiency in capturing relevant case details.
%        \item Precision and recall metrics for the RAG component's retrieval of relevant legal sources and generation of appropriate legal strategies.
%        \item User experience surveys to measure the accessibility, clarity, and actionability of the communicated legal guidance.
%        \item Expert evaluation by legal professionals specializing in gender-based cases to assess the substantive quality of the generated legal strategies.
%    \end{itemize}

\section{Scope}
\begin{itemize}
    \item Quantitative assessment of the chatbot's ability to create legal cases semantically similar to that of legal practitioners.
    \item User experience surveys to measure the accessibility, clarity among other aspects of the communicated legal guidance.
\end{itemize}

\section{Products and Publications}
The products developed within the scope of this research are:

1. A research paper titled ``A Scalable Framework for Legal Text Understanding in Regulatory and Financial Contexts'' has been accepted at the Joint Workshop of FinNLP, FNP, and LLMFinLegal 2025 in Abu Dhabi~\cite{martinez-etal-2025-scalable}. 
The paper presents our findings on using Information Retrieval for legal information extraction and the effectiveness of LLM-curated data.

2. The complete implementation of the legal assistance chatbot is available on GitHub at 
\url{https://github.com/smartinez1/thesis_code}. This includes the conversational agent, RAG implementation, 
and the evaluation framework. The system can be accessed through a web interface at \url{https://legal-assistant-frontend-411325515644.us-central1.run.app/ }.

3. A knowledge base of Colombian legal documents related to gender-based cases has been compiled and processed for RAG applications.

All resources are publicly available in order to facilitate further research in legal assistance systems.
presentation
\endinput