% $Id:abstract.tex  $
% !TEX root = main.tex

\chapter{Abstract}
The intersection of artificial intelligence (AI) and 
the legal field has garnered significant attention in 
recent years, particularly with advancements in technologies 
such as Large Language Models (LLMs). 
These innovations hold transformative potential for 
global legal systems by democratizing access to legal 
information and translating complex legal jargon into 
comprehensible language for non-experts. 
In Colombia, where opaque legal processes and 
limited public accessibility hinder justice, 
AI-driven tools present a promising avenue for enhancing 
legal literacy and equitable access to legal resources.

This study investigates recent developments in AI applications 
for the legal domain, emphasizing the capabilities of LLMs, 
legal chatbots, obligation mining, and retrieval-augmented generation (RAG) 
to bridge the gap between legal systems and the general public. 
To support this research, a synthetic dataset of legal conversations 
was developed through simulated interactions between lawyers and users, 
curated in collaboration with legal experts. 
Furthermore, we propose a novel agent-based architecture designed to: 
(1) ensure comprehensive extraction of case details from user interactions, 
(2) systematically construct legal cases from this information, 
and (3) integrate contextually relevant legal frameworks via RAG to 
generate accurate, actionable guidance for users.

By combining structured dialogue design, expert-validated data, 
and retrieval-augmented reasoning, this approach aims to empower 
individuals to navigate legal challenges efficiently while maintaining 
alignment with jurisdictional regulations. The findings stress the 
viability of AI in fostering transparency and accessibility within complex legal ecosystems. 
\endinput

